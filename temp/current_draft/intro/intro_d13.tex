
\section{Introduction}
%\textcolor{red}{As you yourself have noted, this is very similar to GRL’s introduction and needs to be re-written.}\\
%\textcolor{red}{We will need to include a paragraph (w relevant literature) discussing how shearing changes nonlinearity.}\\

\paragraph{} Dynamic stresses associated with energy production and waste water sequestration (injection, pumping, and transport of supercritical $H_{2}O$--$CO_{2}$ fluids) are particularly concerning as they are known to induce seismicty [Healy et al., 1968; Raleigh et al., 1976; Simpson et al., 1988; Sminchak and Gupta, 2003; McNamara et al., 2015; McGarr et al., 2015; Walsh and Zoback, 2015]. \cite{rtmQ,migExt}

\paragraph{} Though dynamic stressing could be beneficial for enhanced permeability, it presents a significant risk associated with fault reactivation, reservoir seal damage, and accelerated deformation. Therefore, it is important to understand how fluid injection influences the hydromechanical properties of rocks and fractures. We have performed careful laboratory experiments to investigate the connection between fluid flow and elastic nonlinearity (i.e. stress is not proportional to strain) of fractured media.

\paragraph{} Dynamic stressing of the local subsurface associated with seismicity, drilling, hydraulic injection cause transient changes in permeability. These perturbations cause significant changes in the local stress field and consequently the poromechanical properties of the local subsurface. Empirical evidence from the field and laboratory show that earthquakes and subsequent seismic waves cause transient changes in rock stiffness in the proximity of faults. Specifically, measurements of a sudden decrease in seismic wave velocity from co-seismic softening and a post-seismic relaxation of the rock stiffness, a logarithmic recovery in time. Scaling down to laboratory studies, it has been shown that by using dynamic acousto-elastic testing, ultrasonic wave velocity (analog for field-scale seismometers) recorded decreased during dynamic stressing and then recovered after dynamic stressing ended with a logarithmic form. Dynamic perturbations with strain on the order of $10^{-6}$ cause a transient decrease in stiffness in nonlinear elastic materials. 
%\textit{It would be nice to mention any studies b/w lab- and field-scale.} 

\paragraph{} Anthropogenic activities on the field-scale such as drilling, wastewater storage, hydraulic fracturing result in considerable deformation of reservoir rocks. Changing the hydromechanical properties by dynamic stressing from fluid injection are likely to present hazards in the form of fault reactivation and reservoir seal damage. Evidence of this type of stressing inducing regional seismicity is rich and numerous (\textit{more details and citations}). Despite the hazards, dynamic stressing of the subsurface may result in enhanced permeability, consequently greater energy recovery. 
Seismic and anthropogenic sources of dynamic perturbation both change rock stiffness and permeability in similar ways, which suggests there is a physical mechanism that relates the nonlinear stiffness and poromechanical properties of fractured rock.

\paragraph{} Nonlinear response is sensitive to many fracture properties: geometry, flow pathways, asperity compliance, and friction. Currently, literature for investigating the relationship b/w elastodynamic and poromechanical and subsequent recovery is quite limited. We present the results from sophisticated well-controlled laboratory experiments in which we combine the analysis of nonlinear elastodynamic and hydraulic data. 

\paragraph{} It is expected that the nonlinear behavior of rocks is sensitive to fine features such as fracture aperture (i.e. flow pathways, asperity stiffness). In order to fully under-stand the ramifications of dynamic stressing in the subsurface, we need elucidate the relationship between the elastodynamic and hydro-mechanical properties of fractured rocks. 