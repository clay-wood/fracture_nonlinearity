
\section{Conclusions}
\paragraph{}
We present observations on the coupling between elastic stiffness and permeability transients due to dynamic stressing. Understanding this relationship can help illuminate key processes governing flow in fractured reservoirs, energy production and subsurface waste disposal. Our experiments reveal a correlation between changes in permeability and wave velocity due to dynamic stressing via both normal stress and pore pressure oscillations. Within the range of our experimental parameters, oscillations of larger-amplitude and higher frequency tend to be more effective in increasing permeability. Also, pore pressure oscillations are more effective in permeability enhancement than normal stress oscillations of the same amplitude and frequency. Our observations suggest that dynamic stressing is more likely to enhance the permeability of fractured rocks that exhibit greater elastic nonlinearity.
Aperture change and unclogging are candidate mechanisms to explain our observations. The former is likely to be dominant during normal stress oscillations while the latter appears to dominate during pore pressure oscillations. Future experiments on pre-fractured and saw-cut samples of known roughness will quantify the role of aperture change for changes in dynamic stiffness and permeability. We will also look for fines in the downstream effluent before and after dynamic stressing in order to better understand the role of unclogging in promoting permeability enhancement.